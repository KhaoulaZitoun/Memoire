\chapter{Des exigences vers les tests et vice versa}
\label{chap:secondchapitre}

\section{Un DSL pour modéliser le domaine}

Nous avons vu dans le premier chapitre de ce mémoire que le logiciel Cocumber permettait de rédiger des scénarios de tests fonctionnels avec un DSL qui s'apparente à un langage naturelle. Ceci permet donc la compréhension de toutes les parties prenantes de ces scénarios tout en impliquant le domaine. D'autre part, l'outil permet à partir de ce DSL de scénario de générer les tests fonctionnels associé.

Nous souhaitons conserver cette base pour apporter une solution à notre problématique.  

\section{Générer les tests}

\subsection{Génération des méthodes}
\subsection{Rédaction des commentaires}

\section{Analyser les commentaires}

\section{Critique de la solution}